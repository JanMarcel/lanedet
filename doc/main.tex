\documentclass{article}
\usepackage[ngerman]{babel}
\usepackage{graphicx}
\usepackage{subcaption}
\usepackage{acro}
\usepackage{pdfpages}

\title{Echtzeitfähige Deep-Learning-basierte Spurerkennung}
\author{Tim Alkofer, Jan-Marcel Schmidt}
\date{Oktober 2024} %Todo

\begin{document}
    \pagenumbering{Roman}
    
    \maketitle
    
    \begin{tabbing}
    \hspace{5em} \= \\
        Zeitraum: \> 04.2024-10.2024 \\ %Todo
        Hochschule: \> HTWG Konstanz \\
        Rahmen: \> Teamprojekt Autombilinformationstechnik \\
        Betreuer: \> Prof. Dr. Christopher Knievel \\
        Abgabe: \> 26.03.2024 \\ %Todo
    \end{tabbing}

    \clearpage
    \tableofcontents
    \clearpage
    \listoffigures
    \clearpage
    \bibliographystyle{plain}
    \bibliography{literatur}
    % \clearpage
    % \printacronyms
    \clearpage
    \pagenumbering{arabic}

    \section{Einleitung}
        Im Rahmen des Studiums Automobilinformationstechnik an der HTWG Konstanz wird im sechsten Semester ein Teamprojekt durchgeführt.
        Im folgenden soll das Projekt \textit{Echtzeitfähige Deep-Learning-basierte Spurerkennung} vorgestellt werden.

        \subsection{Ausgangssituation}
            % Auto beschrieben
            % bisherige Spurerkennung beschreiben
            % probleme: Lichtverhältnisse, Schatten
            % am besten quantifizieren

        \subsection{Aufgabenstellung}
            % siehe OneNote
            % 
    \section{Umsetzung}
        % Als grundlage diente repository x/y
        \subsection{Recherche}
            % 
        \subsection{Installation}
            % station
        \subsection{Datengrundlage}
            % --> auswahl aufgrund von bester subjektiver performance
            % nachweise für daten
        \subsection{Performance}
            % --> performance der modelle auf Labor PC's
            % --> performance der Modelle auf altem Jetson

            % Konvertierungsversuche zu anderen Formaten (erfolglos)

            % Performance auf neuem Jetson
        \subsection{Steuerung}
            % alte Steuerung
            % Projekt während Autonomes Fahren
            % neue Steuerung
    \section{Fazit}
        % Ergebnisse
        % Probleme
        % Ausblick


\end{document}